\documentclass[11pt, a4paper, dvipdfmx]{jsarticle}
\usepackage{amsmath}
\usepackage{amsthm}
\usepackage[psamsfonts]{amssymb}
\usepackage{color}
\usepackage{ascmac}
\usepackage{amsfonts}
\usepackage{mathrsfs}
\usepackage{amssymb}
\usepackage{graphicx}
\usepackage{fancybox}
\usepackage{enumerate}
\usepackage{verbatim}
\usepackage{subfigure}
\usepackage{proof}
\usepackage{listings}
\usepackage{otf}

\theoremstyle{definition}

%
%%%%%%%%%%%%%%%%%%%%%%
%ここにないパッケージを入れる人は,必ずここに記載すること.
%
%%%%%%%%%%%%%%%%%%%%%%
%ここからはコード表です.
%

\newtheorem{Axiom}{公理}[section]
\newtheorem{Definition}[Axiom]{定義}
\newtheorem{Theorem}[Axiom]{定理}
\newtheorem{Proposition}[Axiom]{命題}
\newtheorem{Lemma}[Axiom]{補題}
\newtheorem{Corollary}[Axiom]{系}
\newtheorem{Example}[Axiom]{例}
\newtheorem{Claim}[Axiom]{主張}
\newtheorem{Property}[Axiom]{性質}
\newtheorem{Attention}[Axiom]{注意}
\newtheorem{Question}[Axiom]{問}
\newtheorem{Problem}[Axiom]{問題}
\newtheorem{Consideration}[Axiom]{考察}
\newtheorem{Alert}[Axiom]{警告}

%%%%%%%%%%%%%%%%%%%%

\newtheorem*{Axiom*}{公理}
\newtheorem*{Definition*}{定義}
\newtheorem*{Theorem*}{定理}
\newtheorem*{Proposition*}{命題}
\newtheorem*{Lemma*}{補題}
\newtheorem*{Example*}{例}
\newtheorem*{Corollary*}{系}
\newtheorem*{Claim*}{主張}
\newtheorem*{Property*}{性質}
\newtheorem*{Attention*}{注意}
\newtheorem*{Question*}{問}
\newtheorem*{Problem*}{問題}
\newtheorem*{Consideration*}{考察}
\newtheorem*{Alert*}{警告}
\renewcommand{\proofname}{\bfseries Proof}

%%%%%%%%%%%%%%%%%%%%%%
%%


%英語で定義や定理を書きたい場合こっちのコードを使うこと.

\newtheorem{Axiom+}{Axiom}[section]
\newtheorem{Definition+}[Axiom+]{Definition}
\newtheorem{Theorem+}[Axiom+]{Theorem}
\newtheorem{Proposition+}[Axiom+]{Proposition}
\newtheorem{Lemma+}[Axiom+]{Lemma}
\newtheorem{Example+}[Axiom+]{Example}
\newtheorem{Corollary+}[Axiom+]{Corollary}
\newtheorem{Claim+}[Axiom+]{Claim}
\newtheorem{Property+}[Axiom+]{Property}
\newtheorem{Attention+}[Axiom+]{Attention}
\newtheorem{Question+}[Axiom+]{Question}
\newtheorem{Problem+}[Axiom+]{Problem}
\newtheorem{Consideration+}[Axiom+]{Consideration}
\newtheorem{Alert+}{Alert}

%commmand

\newcommand{\N}{\mathbb{N}}
\newcommand{\Z}{\mathbb{Z}}
\newcommand{\Q}{\mathbb{Q}}
\newcommand{\R}{\mathbb{R}}
\newcommand{\C}{\mathbb{C}}
\newcommand{\W}{{\cal W}}
\newcommand{\cS}{{\cal S}}
\newcommand{\Wpm}{W^{\pm}}
\newcommand{\Wp}{W^{+}}
\newcommand{\Wm}{W^{-}}
\newcommand{\p}{\partial}
\newcommand{\Dx}{D_{x}}
\newcommand{\Dxi}{D_{\xi}}
\newcommand{\lan}{\langle}
\newcommand{\ran}{\rangle}
\newcommand{\pal}{\parallel}
\newcommand{\dip}{\displaystyle}
\newcommand{\e}{\varepsilon}
\newcommand{\dl}{\delta}
\newcommand{\pphi}{\varphi}
\newcommand{\ti}{\tilde}
\newcommand{\X}{\mathcal{X}}
\newcommand{\A}{\mathcal{A}}
\newcommand{\Borel}{\mathcal{B}}
\newcommand{\E}{\mathbb{E}}
\renewcommand{\P}{\mathbb{P}}
\newcommand{\Prob}{(\Omega, \mathcal{F}, \P)}
\newcommand{\MDP}{(\X, \A, P_{0})}
\title{速習 強化学習}
\author{}
\date{}
\begin{document}
\maketitle
\section{マルコフ決定過程}
完備な確率空間を$\Prob$とする.
\begin{Definition+}(可算MDP)\\
    可算な状態集合を$\X$, 可算な行動集合を$\A$とする. この時遷移確率カーネル$P_{0}:\Borel(\X\times\R)\times\X\times\A\to [0, 1]$を
    \begin{align*}
        P_{0}(U, x, a) = P_{0}(U~|~x, a)
    \end{align*}
    と定める. ここで$P_{0}$をは$\X\times\R$上の確率測度である. この時$(\X, \A, P_{0})$を可算MDPという.
\end{Definition+}
\begin{Definition+}(状態遷移確率カーネル)\\
    $(\X, \A, P_{0})$を可算MDPとする. 状態遷移確率確率カーネル$P:\X\times\A\times\X\to [0, 1]$を
    \begin{align*}
        P(x, a, y) = P_{0}(\{y\}\times\R~|~x, a)
    \end{align*}
    と定める.
\end{Definition+}    

\begin{Definition+}(即時報酬関数, 即時報酬)\\
    $(\X, \A, P_{0})$を可算MDPとする. 
    即時報酬と呼ばれる確率変数$R_{(x, a)}:\Omega\to\X$を考える. ただし$(x, a)\in\X\times\A$であり$R_{(x, a)}$は$(x, a)$に依存する.
    即時報酬が$(Y_{(x, a)}, R_{(x, s)})\sim P_{0}(\cdot|x, a)$である時,(ただし$Y_{(r, a)}:\Omega\to\A$は確率変数)即時報酬関数$r:\X\times\A\to\R$を   
    \begin{align*}
        r(x, a) = \E[R_{(x, a)}]
    \end{align*}    
    と定める.
\end{Definition+}

\begin{Proposition+}
    任意の$x\in\X, a\in\A$について, 確率1で$|R_{(x, a)}| < M$であるならば
    \begin{align*}
        \|r\|_{\infty} = \sup_{(x, a)\in\X\times\A}|r(x, a)|<M
    \end{align*}    
    が成立する.
    \begin{proof}
        \begin{align*}
            r(x, a) = \int_{\Omega}r(x, a) < \int_{\Omega}M d\P = M\times\P(\Omega) = M
        \end{align*}
        これが任意の$(x, a)\in\X\times\A$について成立するので
        \begin{align*}
            \|r\|_{\infty} = \sup_{(x, a)\in\X\times\A}|r(x, a)|<M
        \end{align*}    
        が成り立つ. 
    \end{proof}
\end{Proposition+}

マルコフ決定過程は逐次的な意思決定問題を記述する空間である. $\MDP$を可算MDPとする.
$t\in\N$とし$X_{t}:\Omega\to\X, A_{t}:\Omega\to\A$を確率変数とする. 
この選択された行動が実行されると, システムにそれを反映した行動が生じる.
\begin{align*}
    (X_{t + 1}, R_{t + 1})\sim P_{0}(\cdot|X_{t}, A_{t})
\end{align*}
ここで, 任意の$x, y\in\X$, $a\in\A$について$\P(X_{t + 1} = y~|X_{t} = x, A_{t} = a) = P(x, a, y)$である.
また, 任意の$w\in\Omega$に対して$\E[R_{t+1}|X_{t}(w), A_{t}(w)] = r(X_{t}(w), A_{t}(w))$である.

\begin{Definition+}(行動則)\\
    行動を選択するルールのことを行動則という.
\end{Definition+}
行動則と確率的に決定される初期状態$X_{0}:\Omega\to\X$によって, 状態-行動-報酬の系列$((X_{t}, A_{t}, R_{t + 1};t\geq 0 ))$が確率的に
決定される. 
$(X_{t + 1}, R_{t + 1})$は$(X_{t + 1}, R_{t + 1})\sim P_{0}(\cdot|X_{t}, A_{t})$より, $(X_{t}, A_{t})$に依存している.

\begin{Definition+}(報酬)\\
    $0\leq \gamma\leq 1$を割引率とする. 報酬$R$を以下のように定義する.
    \begin{align*}
        R = \sum_{t = 0}^{\infty} \gamma^{t} R_{t + 1}
    \end{align*}
\end{Definition+}
割引率$\gamma$が0以上1未満で上で定義される報酬を持つ$MDP$を割引報酬MDPと呼び, 割引率が1の時の報酬を持つMDPは割引なしと呼ばれる.

\end{document}
